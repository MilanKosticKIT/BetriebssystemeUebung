\documentclass[12pt,a4paper]{article}
\usepackage[latin1]{inputenc}
\usepackage{amsmath}
\usepackage{amsfonts}
\usepackage{amssymb}
\usepackage{graphicx}
\usepackage{listings}
\usepackage{xcolor}
\usepackage{comment}
\renewcommand{\contentsname}{Inhaltsverzeichnis}

\begin{document}
\tableofcontents
\newpage
\section{Einleitung}
\subsection{Ziel}
Als Erg�nzung zu den Inhalten der Vorlesung "Betriebsysteme" und zur Vertiefung des Systemnahen Programmierens unter C++, wurde das Dateisystem MyFs erstellt.
\subsection{Technische Umgebung}
Da Dateisystemverwaltung auf Kernel Ebene stattfindet, wurde File System In User Space (Fuse) als Schnittstelle zwischen dem Kernel und MyFs benutzt. Somit stand auch das Zielbetriebsystem fest, es wurde Linux aufgrund der besseren Handhabung von Fuse unter Linux.
Entwickelt wurde das Dateisystem mit hilfe von XCode unter MacOS in C++.
Zur Versionsverwaltung und zur Projektverwaltung wurde GitHub benutzt.
\subsection{Struktur der Containerdatei}
\label{subsec:struktur}
Die Containerdatei wurde intern in 5 Abschnitte unterteilt.
\\ 
\begin{enumerate}
\item Superblock  $\Rightarrow$ \textit{Informationen �ber das Dateisystem} 
\item DMAP $\Rightarrow$ \textit{�bersicht freier Datenbl�cke}
\item FAT	$\Rightarrow$ \textit{Verlinkung von Datenbl�cken zu Dateien}  
\item Root $\Rightarrow$ \textit{Informationen �ber Dateien} 
\item Datenbl�cke $\Rightarrow$ \textit{Dateien in MyFs geteilt in Bl�cke} 
\end{enumerate}
\subsection{Projektstruktur}
Bei der Projektplanung haben wir uns an dem MVC Pattern orientiert.
Wir haben getrennte Klassen f�r die einzelnen Abschnitte (siehe \ref{subsec:struktur}), welche die Logik/Modell stellen.
Dazu gibt es die Klasse "FilesystemIO" welche Methoden f�r das Schreiben in die Containerdatei, sowie das Lesen daraus zur Verf�gung stellt und in dem Pattern als View angesehen werden kann.
Objekte dieser Klassen werden von einer �bergeordneten Klasse "MyFS" erzeugt.
Diese ruft Methoden aus den Klassen auf, um die Ausf�hrung des Programms zu steuern und kann als Control in dem MVC Pattern angesehen werden.
\newpage
\section{Implementierung}
\subsection{Superblock}
Der Superblock enth�lt wichtige Informationen �ber das Dateisystem an sich. In ihm werden sowohl Informationen die f�r das Betriebssystem und den Benutzer relevant sind wie Gr��e des Dateisystems, der freie Speicherplatz und maximale Speichergr��e. Als auch f�r den Betrieb des Dateisystems wichtige Daten wie die jeweiligen Startadressen der Bestandteile und die Gr��e dieser. Von der Implementierung ist es eine Struct mit 16 bit unsigned Integer werten.


\subsection{DMAP}
Die DMAP ist dazu da, einen �berblick zu schaffen, welche Datenbl�cke frei sind.
Im Endeffekt handelt es sich um einen Integer Array dessen L�nge, der Anzahl vorhandenen Datenbl�cke im Dateisystem entspricht. Dabei bedeutet eine "0" an Stelle x im Array, dass der x-te Datenblock frei ist, dementsprechend eine "1", dass der Datenblock belegt ist.  
\\
Die Klasse DMAP stellt 5 �ffentliche Methoden, es k�nnen Bl�cke als belegt und frei gesetzt werden.
Au�erdem gibt die Methode "getFreeBlock" den n�chsten freien Block zur�ck. Wir haben uns dazu entschlossen, den Wert f�r den n�chsten freien Block jederzeit im Hintergrund als private Variable zu halten und erst nachdem dieser Block beschrieben wurde, den neuen n�chsten freien Block zu berechnen.
F�r die Initialisierung, wurden zwei Methoden implementiert "getAll" welchen den kompletten DMAP Array zur�ckgibt, und "setAll" welche den DMAP Array mit dem �bergebenen Array �berschreibt. Diese zwei Methoden werden benutzt, um die komplette DMAP in die Containerdatei auf der Festplatte zu schreiben bzw. sie wiederherzustellen.

\subsection{FAT}
Die Aufgabe der FAT ist die Verbindung zwischen einzelner Datenbl�cke zu einer Datei herzustellen.



\begin{comment}
\lstinputlisting
[caption={Superblock}
\label{lst:superblock},
captionpos=b,language=C++,firstline=29,lastline=41]
{../includes/myfs-structs.h}


\lstinputlisting
[caption={DMAP}
\label{lst:dmap},
captionpos=b,language=C++, basicstyle=\ttfamily,
keywordstyle=\color{blue}\ttfamily,
stringstyle=\color{purple}\ttfamily,
commentstyle=\color{gray}\ttfamily,
]%
{../src/dmap.cpp}
\end{comment}

\end{document}